\documentclass[]{article}

%opening
\title{Physics of Motion: Weekly Problems 1}
\author{Casey Farren-Colloty}
\date{}
\usepackage{amsmath,amsfonts,amssymb,amsthm}
\usepackage{braket}
\usepackage[margin=1.0in]{geometry}
\usepackage{bbold}
\usepackage{physics}
\begin{document}

\maketitle

\section{Problem 1}
A particle travels along the line $y=2x+1$
with uniform speed $\va{V}$. Find the components of its velocity parallel to the axes of x and y.
\vspace{5mm}

\textbf{Solution}\\
The x and y components of a vector $\va{V}$ are denoted as $\va{V_{x}}$ and $\va{V_{y}}$ respectively.
\\
It follows from Pythagoras' Theorem that the x and y components of the vector $\va{V}$ could be written in the forms
\begin{equation}
	\va{V_{x}}=\va{V}\cos\theta
\end{equation}
\begin{equation}
	\va{V_{y}}=\va{V}\sin\theta
\end{equation}
Let $\theta$ = The angle between the line $y$ and the x axis.\\
Using $tan\theta = m$ where m = the slope of the line $y=2x+1$ in the form $y=mx+c$. We find that:\\
	\[
	\tan\theta = 2
	\]
		\[
	\arctan(2) = \theta
		\]
	\begin{equation}
	\theta \approx\ 63.435^\circ
\end{equation}\\
From inserting the value of $\theta$ from Eq.3 into the values for $\va{V_{x}}$ and $\va{V_{y}}$ from Eq.1 and Eq.2 respectively:
\[
\va{V_{x}}=\va{V}\cos63.435^\circ
, \va{V_{y}}=\va{V}\sin63.435^\circ
\]
\section{Problem 2}
The velocities of the two particles at time \textit{t} are $2t\va{i}+12\va{j}$ and $4\va{i}+(3-2t)\va{j}$, respectively. Find the instant when the particles are moving in perpendicular directions.
\vspace{5mm}

\textbf{Solution}\\
Let $\va{A}=2t\va{i}+12\va{j}$ and $\va{B}=4\va{i}+(3-2t)\va{j}$. We can rewrite these vectors in the form:\\ 
\begin{equation}
	\va{A}=\begin{pmatrix}
	2t & 12 & 0
\end{pmatrix}
\end{equation}
\begin{equation}
\va{B}=\begin{pmatrix}
4 & 3-2t & 0
\end{pmatrix}
\end{equation}
To find the time in which the vectors are moving in perpendicular directions, we must find the time $t$ such that the dot product 
\begin{equation}\va{A}\cdot\va{B}=0
\end{equation}
\begin{align*}
\begin{pmatrix}
	2t & 12 & 0
\end{pmatrix}
\cdot
\begin{pmatrix}
	4 & 3-2t & 0
\end{pmatrix} &= 2t\cdot4+12\cdot(3-2t)\\
&=8t+36-24t\\
&=36-16t
\end{align*}
\begin{equation}
	\va{A}\cdot\va{B}=36-16t
\end{equation}
\[Eq. 6 = Eq. 7\]
\[
0=36-16t\]\[
16t=36\]\[
t=\frac{36}{12}
\]
\begin{equation}
	t=\frac{9}{4}
\end{equation}
From Equation 8 we can see that the two particles are moving in perpendicular directions when $t=\frac{9}{4}$units of time.
\end{document}
