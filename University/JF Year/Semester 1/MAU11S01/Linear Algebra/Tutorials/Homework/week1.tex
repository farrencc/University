\documentclass[]{article}

%opening
\title{}
\author{Casey Farren-Colloty}
\date{}
\usepackage{amsmath,amsfonts,amssymb,amsthm}
\usepackage{braket}
\usepackage[margin=1.0in]{geometry}
\usepackage{bbold}
\usepackage{physics}
\usepackage{graphicx}

\begin{document}

\maketitle

\section{Problem 1}
Consider the two-dimensional plane with the usual coordinate axes $x$ and $y$
\begin{figure}
	\graphicspath{{"C:/Users/casey/OneDrive/Desktop/University/JF Year/Semester 1/MAU11S01/Linear Algebra/Tutorials/Homework/Diagrams/week1vectordiagram.png"}}
	\includegraphics[width=50mm]{C:/Users/casey/OneDrive/Desktop/University/JF Year/Semester 1/MAU11S01/Linear Algebra/Tutorials/Homework/Diagrams/week1vectordiagram.png}
	\caption{Vectors $\vec{P}$ and $\vec{Q}$}
\end{figure}
\vspace{5mm}

\textbf{Solution}\\

For solutions for Problem 1 (a),(b), and (d) see Figure 1\\

(c)\\
 \begin{equation}
\vec{P} = 2 \vec{i} + 3\vec{j}
\end{equation}
\begin{equation}
\vec{Q} = 5\vec{i} +1\vec{j}
\end{equation}
(e)\\
\begin{align*}
\abs{\abs{\vec{Q}-\vec{P}}} & = \abs{\abs{(5\vec{i} +1\vec{j}) - (2 \vec{i} + 3\vec{j})}}\\
& = \abs{\abs{(5-2)\vec{i}+(1-3)\vec{j}}}\\
& = \abs{\abs{3\vec{i}-2\vec{j}}}\\
& = \sqrt{(3)^2+(-2)^2}\\
& = \sqrt{13}
\end{align*}
\begin{equation}
\implies \abs{\abs{\vec{Q}-\vec{P}}} = \sqrt{13}
\end{equation}\\
(f)\\
\[
\abs{PQ} = \sqrt{(5-2)^2+(1-3)^2}
\]\begin{equation}
\abs{PQ} = \sqrt{13}
\end{equation}


\section{Problem 2}
For $v = -2i + 6j$ and $w = 10j -3i$,\\
(a) Calculate $\abs{\abs{v+w}}$\\
(b) What are the coordinates of the points in the plan with position vectors v and w?\\
(c) Calculate $v\cdot w$\\
(d) Show that $\abs{\abs{v+w}} \leq \abs{\abs{v}} + \abs{\abs{w}}$\\
(e) Calculate $\cos\theta$ where $\theta$ is the angle between v and w
\vspace{5mm}

\textbf{Solution}\\

(a)\begin{align*}
\abs{\abs{v+w}} & = \abs{\abs{(-2\vec{i}+6\vec{j}) + (-3\vec{i}+10\vec{j})}}\\
& = \abs{\abs{(-2-3)\vec{i} + (6+10)\vec{j}}}\\
& = \abs{\abs{-5\vec{i}+16\vec{j}}}\\
& = \sqrt{(-5)^2 + (16)^2}\\
& = \sqrt{281}
\end{align*}
\begin{equation}
	\abs{\abs{v+w}} = \sqrt{281}
\end{equation}\\
(b) $v = (-2,6) w = (-3,10)$
(c) \begin{equation}
	\vec{v}=\begin{pmatrix}
	-2 & 6 & 0
\end{pmatrix}
\end{equation}
\begin{equation}
	\vec{w}=\begin{pmatrix}
		-3 & 10 & 0
	\end{pmatrix}
\end{equation}
\[
v\cdot w = \begin{pmatrix}
	-2 & 6 & 0
\end{pmatrix} \cdot
\begin{pmatrix}
	-3 & 10 & 0
\end{pmatrix}
\]\[
v \cdot w = (-2)(-3) + (6)(10)
\]
\begin{equation}
	v \cdot w = 66
\end{equation}
(d)
\[
\abs{\abs{v+w}} \leq \abs{\abs{v}} + \abs{\abs{w}}
\]\begin{align*}
\text{(5)... }\sqrt{281} & \leq \abs{\abs{v}} + \abs{\abs{w}}\\
& \leq \abs{\abs{-2\vec{i} +6\vec{j}}} + \abs{\abs{-3\vec{i} + 10\vec{j}}}\\
& \leq \sqrt{(-2)^2 + (6)^2} + \sqrt{(-3)^2 + (10)^2}\\
& \leq \sqrt{40} + \sqrt{109}
\end{align*}
\begin{equation}
	\sqrt{281} \approx 16.763 \leq 2\sqrt{10} + \sqrt{109} \approx 16.765 \text{ } \blacksquare
\end{equation}
(e)\[
\theta = \arctan(\frac{-3\vec{i}}{10\vec{j}}) - \arctan(\frac{-2\vec{i}}{6\vec{j}})
\]\[
\theta \approx 1.736
\]\begin{equation}
	\implies \cos\theta = \cos(1.736) \approx 1
\end{equation}



\end{document}
